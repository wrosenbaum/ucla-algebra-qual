\section{Field Extensions}

We consider the category of $\Fields$. The objects in this category are fields, and the morphisms are ring (field) homomorphisms. Let $F$ and $K$ be fields, and suppose there is a homomorphism
\[
	\alpha : F \too K.
\]
Then $\ker \alpha \subset F$  is an ideal and $1 \notin \ker \alpha$, implying that $\ker \alpha$ is trivial. Therefore $\alpha$ is injective and $F \simeq \alpha(F) \subset K$. In particular, every field homomorphism is injective. 

\begin{dfn}
	Let $K$ be a field and $F$ a subfield. The we say that $F$ is a \emph{field extension} of $F$ and write $K / F$ or $F \subset K$ or
	\[
		\begin{psmatrix}[rowsep=0.5cm, colsep=0.5cm, linewidth=1pt]
			K\\
			F
			\psset{arrows=-, nodesep=3pt}
			\everypsbox{\scriptstyle}
			\ncline{2,1}{1,1}
		\end{psmatrix}
	\]
	Notice that if $K / F$ is an extension, then $K$ has the structure of a vector space over $F$. Then
	\[
		\dim_F(K) = [K : F]
	\]
	is the \emph{degree} of $K / F$.
\end{dfn}

\begin{eg}
	\begin{enumerate}
		\item $F \subset F$ is the trivial extension, $\set 1$ is a basis so $[F:F] = 1$.
		\item $[\C, \R]$ has basis $\set{1,i}$ so $[\C, \R] = 2$.
		\item $[\R : \Q] = \infty$ ($\R$ is not countable).
	\end{enumerate}
\end{eg}

\begin{prop}
	Let $L / K / F$ be field extensions. Then
	\[
		[L:F] = [L:K]\cdot [K:F].
	\]
\end{prop}
\begin{proof}
	Let $\set{x_i}_{i\in I}$ be a basis for $K / F$ and $\set{y_j}_{j \in J}$ be a basis for $L / K$. We claim that $\set{x_i y_j}$ is a basis for $L / F$. First, we show linear independence. Suppose we have
	\[
		\sum_{i,j} a_{ij} x_i y_j = 0.
 	\]
	Then
	\begin{align*}
		&\sum_j \paren{\sum_i a_{ij} x_i}y_j = 0\\
		&\implies \sum_{i} a_{ij} 0 \quad \text{for all } j\\
		&\implies a_{ij} = 0 \quad \text{for all } i,j.
	\end{align*}
	Therefore, the $x_i y_j$ are independent. To see that they generate $L$, take $z \in L$. Then we can write
	\[
		z = \sum_j v_j y_j = \sum j \sum i a_{ij} x_i y_j = \sum_{i,j} a_{ij} x_i y_j
	\]
	with $a_{ij}$ in $F$. Therefore $\set{x_i y_j}$ generates $L / K$. 
\end{proof}

\begin{cor}
	If $L / F$ is finite then $[L:K]$ and $[K:F]$ are also finite and divide $[L : F]$.
\end{cor}

\begin{dfn}
	Let $K$ be a field and $S \subset K$ a subset. Then form the intersection.
	\[
		\bigcap_{K' \subset K\\ S\subset K'} K'
	\]
	This intersection is the \emph{subfield generated by} $S$. Suppose $K / F$ is a field extension and $S \subset K$. Then $F(S)$ is the subfield of $K$ generated by $S \cup F$. If $S = \set{\alpha_1, \ldots, \alpha_n} \subset K$ then we write
	\[
		F(\alpha_1, \ldots, \alpha_n)
	\]
	for the field generated by $S$ over $F$.
\end{dfn}

\begin{eg}
	Suppose $S = \varnothing$ and $F = \C$. Then the subfield generated by $S$ is $\Q$. Every subfield of $\C$ contains $0$ and $1$, and therefore contains $\Z$. Therefore, the smallest subfield containing $S$ must contain $\Q$ as well.
\end{eg}

\begin{prop}
	\[
		F(\alpha_1, \ldots, \alpha_n) = K' = \set{\frac{f(\alpha_1, \ldots, \alpha_n)}{g(\alpha_1, \ldots, \alpha_n)} \st f,g \in F[X_1, \ldots, X_n],\ g \neq 0}.
	\]
\end{prop}
\begin{proof}
	Note that $K'$ is a subfield of $K$ satisfying $F \subset K' \subset K$. Moreover, $K'$ contains all $\alpha_i$. Therefore $F(\alpha_1, \ldots,\alpha_n) \subset K'$. Conversely, $\alpha_i \in F(\alpha_1, \ldots, \alpha_n)$ implying that
	\[
		\frac{f(\alpha_1, \ldots, \alpha_n)}{g(\alpha_1, \ldots, \alpha_n)} \in F(\alpha_1, \ldots, \alpha_n).
	\]
	Therefore, $K' \subset F(\alpha_1, \ldots, \alpha_n)$.
\end{proof}

We can also define $F[\alpha_1, \ldots, \alpha_n]$ to be the \emph{subring} of $K$ generated by $\alpha_1, \ldots, \alpha_k$ over $F$. Then
\[
	F[\alpha_1, \ldots, \alpha_n] = \set{f(\alpha_1, \ldots, \alpha_n) \st f \in F[X_1, \ldots, X_n]} \subset F(\alpha_1, \ldots, \alpha_n).
\]

\begin{eg}
	Extensions:
	\begin{enumerate}
		\item $\C / \R$, $\R[i] = \C = \R(i)$.
		\item $F[X]$ is the polynomial ring, $F[X] \subsetneq F(X)$.
	\end{enumerate}
\end{eg}

\begin{dfn}
	Let $K/F$ be a field extension, $\alpha \in K$. Then $\alpha$ is \emph{algebraic over} $F$ if there exists a nonzero polynomial $f \in F[X]$ such that $f(\alpha) = 0$. If $\alpha$ is not algebraic, then $\alpha$ is \emph{transcendental over} $F$. We say that $F/K$ is an \emph{algebraic extension} if every $\alpha\in K$ is algebraic over $F$. Otherwise the extension is \emph{transcendental}. 
\end{dfn}

\begin{eg}
	Algebraic and transcendental extensions:
	\begin{enumerate}
		\item $K / F$, $\alpha \in F$ is algebraic over $F$ because $\alpha$ satisfies $X - \alpha = 0$.
		\item $\C$ is algebraic over $\R$ because $a + bi$ is a root of $X^2 - 2 a X + (a^2 + b^2)$.
		\item $\pi$ and $e$ are transcendental in $\R / \Q$. Further, set set of algebraic elements of $\R / \Q$ is countable, so almost ever real number is transcendental.
		\item $L / K / F$,  $\alpha \in L$. Suppose $\alpha$ is algebraic over $F$. Then it is algebraic over $K$.
		\item If $\alpha \in K / F$ is transcendental over $F$, then
		\[
			F[\alpha] \simeq F[X]
		\]
		where $f \mapsto f(\alpha)$ is easily seen to be bijective.
		\item $X \in F(X)$. Then $X$ is transcendental over $F$.
	\end{enumerate}
\end{eg}

\begin{thm}
	Let $\alpha \in K/F$ be algebraic over $F$. Then there exists a unique minimal monic irreducible polynomial $m_\alpha \in F[X]$ such that $m_\alpha(\alpha) = 0$. If $f \in F[X]$ satisfies $f(\alpha) = 0$ then $m_\alpha \divides f$. Further, if $n = \deg(m_\alpha)$ then $1, \alpha, \ldots, \alpha^{n-1}$ form a basis for $F(\alpha)/F$, so $[F(\alpha):F] = n$. Finally, $F[\alpha] = F(\alpha) \simeq F[X] \big / m_\alpha F[X]$.
\end{thm}
\begin{proof}
	Let $I = \set{f \in F[X] \st f(\alpha) = 0} \subset F[X]$. Then $I$ is a nonempty ideal because $\alpha$ is algebraic. Since $F[X]$ is a PID, we have
	\[
		I = m_\alpha \cdot F[X] 
	\]
	where $m_\alpha$ is uniquely determined once we require it to be monic. Further, if $f(\alpha) = 0$ then $m_\alpha \divides f$, which implies the uniqueness of $m_\alpha$.
	
	Let $g: F[X] \to K$ be defined by $g(f) = f(\alpha)$. Then $I = \ker g$ and $\im g = F[\alpha]$. By the first isomorphism theorem, we have
	\[
		F[X] / I \simeq F[\alpha].
	\]
	Since $F[\alpha]$ is a domain, $I$ must be a prime ideal, implying that $m_\alpha$ is irreducible. Since $F[X]$ is a PID, and $I$ is prime, it is also maximal, implying that $F[\alpha]$ is a field, hence $F[\alpha] = F(\alpha)$.
	
	Finally, let $\beta \in F(\alpha) = F[\alpha]$ with $\beta = h(\alpha)$. Then $h \in F[X]$, implying that
	\[
		h = m_\alpha \cdot q + r \quad \text{with} \quad q,r \in F[X] \deg(r) < n.
	\]
	This implies that
	\[
		\beta = g(\alpha) = r(\alpha) = a_0 \cdot 1 + \cdots + a_{n-1} \alpha^{n-1}
	\]
	since $m_\alpha(\alpha) = 0$. Therefore $1, \ldots, \alpha^{n-1}$ generate $F(\alpha) / F$. Note that
	\[
		\dim(F[X] / I) = \deg(m_\alpha) = n
	\]
	implying that $1, \ldots, \alpha^{n-1}$ are linear independent, hence a basis.
\end{proof}

\begin{dfn}
	The polynomial $m_\alpha \in F[X]$ in the previous theorem is the \emph{minimal polynomial of } $\alpha$.
\end{dfn}

\begin{rem}
	If $f \in F[X]$ and $\deg f = 2$ or $3$, then $f$ is irreducible if and only if $f$ has no roots.
\end{rem}

\begin{eg}
	Minimal polynomials:
	\begin{enumerate}
		\item $\C / \R$, $\alpha = i$ then $m_i = X^2 + 1$.
		\item $\Q(\sqrt 3) / \Q$, $\alpha = \sqrt 3$ the $m_\alpha = X^2 - 3$
		\item $p$ a prime, $\xi_p$ a primitive $p$-th root of unity, and consider
		\[
			\Q(\xi_p) / \Q.
		\]
		Then $\xi_p$ satisfies $X^p - 1$, but
		\[
			X^p - 1 = (X-1)(X^{p-1} + X^{p-2} + \cdots + 1)
		\]
		where the second factor is irreducible. Therefore
		\[
			X^{p-1} + X^{p-2} + \cdots + 1 = m_{\xi_p}
		\]
		implying that $[\Q(\xi_p):\Q] = p-1$.
	\end{enumerate}
\end{eg}

\begin{cor}
	Let $\alpha \in K / F$. Then $\alpha$ is algebraic over $F$ if and only if $[F(\alpha) : F] < \infty$. 
\end{cor}
\begin{proof}
	$\Rightarrow:$ $[F(\alpha): F] = \deg(\alpha) < \infty$.
	
	$\Leftarrow:$ Consider the elements $1, \alpha, \ldots, \alpha^n$ with $n \geq [F(\alpha) : F] = \dim_F(F(\alpha))$. Then $\set{1, \ldots, \alpha^n}$ is linearly dependent over $F$ implying that there exist $a_0, \ldots, a_n \in F$ not all zero such that
	\[
		a_0 + \cdots + a_n \alpha^n = 0
	\]
	implying that $\alpha$ is algebraic.
\end{proof}

\begin{cor}
	A finite extension is algebraic.
\end{cor}
\begin{proof}
	Suppose $K/F$ has finite degree, and let $\alpha \in K$. Then
	\[
		F(\alpha) \subset K \implies [F(\alpha):F] \leq [K:F] < \infty.
	\]
	Therefore $\alpha$ is algebraic by the previous corollary.
\end{proof}

\begin{cor}
	Let $\alpha_1, \ldots, \alpha_n \in K/F$ be algebraic over $F$. Then
	\begin{enumerate}
		\item $F(\alpha_1, \ldots, \alpha_n) = F[\alpha_1, \ldots, \alpha_n]$
		\item $[F(\alpha_1, \ldots, \alpha_n) : F] < \infty$
	\end{enumerate}
\end{cor}
\begin{proof}
	We argue by induction on $n$. The case $n=1$ is known. Now we will show that $n-1 \implies n$. For 1, we have
	\begin{align*}
		F(\alpha_1, \ldots, \alpha_n) &= F(\alpha_1, \ldots, \alpha_{n-1})(\alpha_n)\\
		&= F(\alpha_1, \ldots, \alpha_{n-1})[\alpha_n]\\
		&= F[\alpha_1, \ldots, \alpha_{n-1}][\alpha_n]\\
		&= F[\alpha_1, \ldots, \alpha_n].
	\end{align*}
	For 2, notice that both extensions in the tower
	\[
		\begin{psmatrix}[rowsep=0.5cm, colsep=0.5cm, linewidth=1pt]
			F(\alpha_1, \ldots, \alpha_n)\\
			F(\alpha_1, \ldots, \alpha_{n-1})\\
			F
			\psset{arrows=-, nodesep=3pt}
			\everypsbox{\scriptstyle}
			\ncline{1,1}{2,1}
			\ncline{2,1}{3,1}
		\end{psmatrix}
	\]
	are finite, hence $F(\alpha_1, \ldots, \alpha_n) / F$ is finite.
\end{proof}

\begin{thm}
	Let $L/K$ be a field extension and let
	\[
		E = \set{\alpha\in K \st \alpha \text{ algebraic over } F}.
	\]
	Then $E$ is a field.\sidenotemark{In fact, $E$ is the largest algebraic extension of $K$ contained in $L$.}
\end{thm}
\begin{proof}
	Let $\alpha, \beta \in E$. Let $\gamma = \alpha + \beta$, $\alpha\beta$ or $\alpha^{-1}$ ($\alpha \neq 0$). Then we need to show that $\gamma \in E$. But $F(\gamma) \subset F(\alpha, \beta)$  which is a finite extension of $F$. Therefore $[F(\gamma) : F] < \infty$, implying that $\gamma \in E$ so $E$ is indeed a field. 
\end{proof}

\begin{thm}
	Let $L/K$ and $K/F$ be algebraic field extensions. Then $L/F$ is algebraic.\sidenote{\[
		\begin{psmatrix}[rowsep=0.5cm, colsep=1cm, linewidth=1pt]
			L\\
			K\\
			F
			\psset{arrows=-, nodesep=3pt}
			\everypsbox{\scriptstyle}
			\ncline{1,1}{2,1}
			\ncline{2,1}{3,1}
		\end{psmatrix}
	\]}
\end{thm}
\begin{proof}
	Let $\alpha \in L$. Since $\alpha$ is algebraic over $k$, there is some $f = \beta_n X^n + \cdots + \beta_0 \in K[X]$ such that $f(\alpha) = 0$. Let $E = F(\beta_0,\ldots,\beta_n)$. Then $E/F$ is a finite extension since all $\beta_i$ are algebraic over $F$. Then $f \in E[X]$, hence $\alpha$ is algebraic over $E$ so
	\[
		[E(\alpha) : E] < \infty, \quad [E:F] < \infty \implies [E(\alpha):F] < \infty.
	\]
	Since $F(\alpha) \subset E(\alpha)$, we have $[F(\alpha) : F] < \infty$, implying that $\alpha$ is algebraic over $F$.
\end{proof}

\begin{thm}
	Let $f \in F[X]$ be a non-constant polynomial. Then there is an extension $K/F$ with $[K:F]\leq \deg f$ such that $f$ has a root in $K$.
\end{thm}
\begin{proof}
	Let $g$ be an irreducible factor of $f$. Take
	\[
		K = F[X] / g\cdot F[X].
	\]
	Since $g$ is irreducible, $K$ is a field. Let
	\[
		F \too K \quad \text{given by} \quad a \longmapsto a + g\cdot F[X]
	\]
	be the natural embedding so that we may identify $F \subset K$. Then
	\[
		[K : F] = \deg g \leq \deg f.
	\]
	To see that $f$ has a root in $K$, consider $\bar X \in K$. Then
	\[
		g(\bar X) = \bar{g(X)} = 0
	\]
	so $\bar X$ is a root of $g$, hence $\bar X$ is also a root of $f$.
\end{proof}

\begin{cor}
	Let $f \in F[X]$ be a polynomial of degree $n>0$. Then there is a field extension $K/F$ with $[K : F] \leq n!$ such that $f$ is \emph{split}\sidenotemark{Recall that a polynomial is \emph{split} if and only if it can be written as a product of linear factors.} over $K$.
\end{cor}
\begin{proof}
	We argue by induction on $n$. If $n = 1$, then $K =F$, so we are done. Now we will show that $n-1 \implies n$. By the theorem, there exists a field extension $K' / F$ with $[K' : F] \leq n$ such that $f$ has a root $\alpha$ in $K'$. Then write
	\[
		f = (X - \alpha)\cdot g \quad \text{with} \quad g \in K'[X].
	\]
	Then $\deg g = n - 1$, so by induction, there exists $K/K'$ with $[K : K'] \leq (n-1)!$ such that $g$ splits over $K$. Thus
	\[
		[K:F] = [K:K'][K':F] \leq n!
	\]
	so that $f = (X - \alpha)\cdot g$ splits over $K$.
\end{proof}

If $f = a (X-\alpha_1)\cdots(X-\alpha_n)$ over $K$ then $\alpha_1, \ldots, \alpha_n$ are all of the roots of $f$ in $K$. 

\begin{dfn}
	Let $f \in F[X]$ be a non-constant polynomial. Then a field extension $K/F$ is called a \emph{splitting field} of $f$ if
	\begin{enumerate}
		\item $f$ is split over $K$
		\item $K = F(\alpha_1, \ldots, \alpha_n)$ where $\alpha_1, \ldots, \alpha_n$ are \emph{all} of the roots of $f$ in $K$.
	\end{enumerate}
\end{dfn}

To see that splitting fields exist, take an extension $K'/F$ such that $f$ is split over $K'$. (Notice that $[K':F] \leq (\deg f)!$.) Then take $K = F(\alpha_1, \ldots, \alpha_n)$ where $\alpha_i$ are all roots of $f$ in $K'$. Then we have
\[
	f = a(X-\alpha_1) \cdots (X-\alpha_n) \quad \text{over } K
\]
implying that $f$ is split over $K$. Therefore $K$ is a splitting field of $f$, and $[K:F]\leq (\deg f)!$.

\begin{eg}
	Splitting fields:
	\begin{enumerate}
		\item $X^2 - 3$ over $\Q$, then $\Q(\sqrt 3)$ is a splitting field.
		\item $X^n - 1$ over $\Q$, then $X^n-1 = (X-1)(X-\xi_n) \cdots (X - \xi_n^{n-1})$, implying that $\Q(\xi_n)$ is a splitting field of $X^n-1$.
	\end{enumerate}
\end{eg}

\begin{dfn}
	Let $K/F$ and $K'/F'$ be field extensions and $\varphi : F \to F'$ a homomorphism. Then we  call $\psi : K \to K'$ an \emph{extension} of $\varphi$ if $\psi(a) = \varphi(a)$ for all $a \in F$. This is indicated by the following diagram:
	\[
		\begin{psmatrix}[rowsep=0.5cm, colsep=1cm, linewidth=1pt]
			& K'\\
			K & F'\\
			F
			\psset{arrows=-, nodesep=3pt}
			\everypsbox{\scriptstyle}
			\ncline{3,1}{2,1}
			\ncline[arrows=->]{3,1}{2,2}_{\varphi}
			\ncline{2,2}{1,2}
			\ncline[linestyle=dashed,arrows=->]{2,1}{1,2}^{\psi}
		\end{psmatrix}
	\]
\end{dfn}

\begin{rem}
	Given a field homomorphism $\varphi: F \to F'$, we can extend $\varphi$ to a map on polynomials. In particular, suppose
	\[
		f = a_n X^n + \cdots + a_0 \in F[X].
	\]
	The we can define the polynomial
	\[
		\varphi(f) = \varphi(a_n) X^n + \cdots + \varphi(a_0) \in F'[X].
	\]
\end{rem}

\begin{prop}
	Let $K = F(\alpha)$ be a finite field extension of $F$, $f \in F[X]$ the minimal polynomial of $\alpha$, $K' / F'$ a field extension and $\varphi : F \to F'$ a field homomorphism. Then
	\begin{enumerate}
		\item If $\psi : K \to K'$ is an extension of $\varphi$, then $\psi(\alpha)$ is a root of $\varphi(f)$.
		\item For any root $\alpha'$ of $\varphi(f)$ in $K'$, there exists a unique extension $\psi : K \to K'$ such that $\psi(\alpha) = \alpha'$.
	\end{enumerate}
\end{prop}
\begin{proof}
	To prove 1, suppose $f(X) = a_n X^n + \cdots + a_0$. Then we have $a_n \alpha^n + \cdots + \alpha_0 = 0$. Applying $\psi$ gives
	\begin{align*}
		& \psi(a_n)\psi(\alpha)^n + \cdots + \psi(a_0) = 0\\
		&\implies \varphi(a_n)\psi(\alpha)^n + \cdots + \varphi(a_0) = 0\\
		&\implies \varphi(f)(\psi(\alpha)) = 0
	\end{align*}
	so that $\psi(\alpha)$ is a root of $\varphi(f)$.
	
	To prove 2, notice that $\psi$ is uniquely determined because $\alpha$ generates $K$. For existence, define 
	\[
		\psi' : F[X] \to K' \quad \text{by} \quad \psi'(g) = \varphi(g)(\alpha').
	\]
	Then we have $\psi'(f) = \varphi(f)(\alpha') = 0$ because $\alpha'$ is a root of $\varphi(f)$. Since $f \in \ker \psi'$, the map $\psi'$ factors through the ring homomorphism
	\[
		\vartheta : \underset{\simeq F(\alpha) = K}{\underbrace{F[X] \big / f \cdot F[X]}} \too K'.
	\]
	Define the map $\psi = \vartheta \circ \pi$ where $\pi$ is the projection of $F[X]$ to $F[X] / f F[X]$. Then we have $\alpha \mapsto \bar X \mapsto \alpha'$ so that $\psi(\alpha) = \alpha'$ where $\psi$ is an extension of $\varphi$ because for $a \in F$, $a \mapsto \bar a \mapsto \varphi(a) \in F'$.
\end{proof}

\begin{cor}
	Using the same hypotheses as the theorem, there are at most $\deg \alpha$ extensions $\psi : K \to K'$ of $\varphi$.
\end{cor}
\begin{proof}
	The number of extensions is equal to the number of roots of $\varphi(f)$ in $K'$, which is as most the degree of $\varphi(f)$. Then $\deg (\varphi(f)) = \deg f = \deg \alpha$.
\end{proof}

\begin{thm}
	Let $K/F$ be a splitting field of a non-constant polynomial $f \in F[X]$ and $\varphi : F \to F'$ a field isomorphism. Let $K' / F'$ be a splitting field of $\varphi(f) \in F'[X]$. Then there is an isomorphism $\psi: K \to K'$ extending $\varphi$.
\end{thm}
\begin{proof}
	We argue by induction on $n = \deg f$. For the case $n = 1$, we have $K = F$, $K' = F'$ and $\psi = \varphi$. Now we argue that $n -1$ implies $n$. Since $f$ splits over $K$, pick a root $\alpha$ of $f$ in $K$. Let $m_\alpha$ be the minimal polynomial of $\alpha$ over $F$, $m_\alpha \in F[X]$ adn $m_\alpha \divides f$. Then $\varphi(f) \in F'[X]$ and $\varphi(f)$ splits over $K'$. Since wer have $\varphi(m_\alpha) \divides \varphi(f)$, we have that $\varphi(m_\alpha)$ is split over $K'$. Let $\alpha'$ be a root of $\varphi(m_\alpha)$ in $K'$. By the proposition, $\varphi$ extends to a field homomorphism
	\[
		\rho : F(\alpha) \too K' \quad \text{such that} \quad \rho(\alpha) = \alpha'.
	\]
	Then $\im \rho = F'(\alpha')$. Thus, we have the following diagram:
	\[
		\begin{psmatrix}[rowsep=1cm, colsep=1cm, linewidth=1pt]
			& K'\\
			K & F'(\alpha')\\
			F(\alpha) & F'\\
			F
			\psset{arrows=-, nodesep=3pt}
			\everypsbox{\scriptstyle}
			\ncline[arrows=->]{4,1}{3,2}^{\varphi}_{\simeq}
			\ncline[arrows=->]{3,1}{2,2}^{\rho}_{\simeq}
			\ncline{4,1}{3,1}
			\ncline{3,2}{2,2}
			\ncline{3,1}{2,1}
			\ncline{2,2}{1,2}
		\end{psmatrix}
	\]
	Write $f = (X - \alpha)g$ with $g \in F(\alpha)[X]$ and $\deg g = n-1$. Then we have
	\[
		\varphi(f) = (X - \alpha') \rho(g) \quad \text{with} \quad \rho(g) \in F'(\alpha')[X].
	\]
	We can write $K = F(\alpha, \alpha_2, \ldots, \alpha_n)$ where the $\alpha_i$ are roots of $f$. Therefore
	\[
		K = F(\alpha)(\alpha_2, \ldots, \alpha_n)
	\]
	are all roots of $g$, implying that $K / F(\alpha)$ is a splitting field of $g$. Similarly, $K' / F'(\alpha')$ is a splitting field of $\rho(g)$. By induction, there exists an isomorphism $\psi : K \to K'$ extending $\rho$. Since $\rho$ extends $\varphi$, this implies that $\psi$ also extends $\varphi$.
\end{proof}

\begin{dfn}
	Suppose we have the following extensions:
	\[
		\begin{psmatrix}[rowsep=1cm, colsep=0.5cm, linewidth=1pt]
			K & & K'\\
			& F
			\psset{arrows=-, nodesep=3pt}
			\everypsbox{\scriptstyle}
			\ncline{2,2}{1,1}
			\ncline{2,2}{1,3}
			\ncline[arrows=->,linestyle=dashed]{1,1}{1,3}^{\psi}
		\end{psmatrix}
	\]
	Then $\psi$ is an \emph{isomorphism of field extensions} if $\psi(a) = a$ for all $a \in F$. In this case, we say that $\psi$ is an isomorphism \emph{over} $F$ or $\psi$ is an $F$\emph{-isomorphism}.
\end{dfn}

\begin{thm}
	(uniqueness of splitting fields). Let $f \in F[X]$ be a (non-constant) polynomial and let $K/F$ and $K'/F$ be two splitting fields. Then there is an isomorphism $\psi : K \to K'$ over $F$.
\end{thm}

\begin{rem}
	Let $f \in F[X]$ be non-constant. Let $L/F$ be a field extension with $f$ split in $L$. The there exists a unique subfield $K \subset L$ with $F \subset K$ such that $K / F$ is a splitting field of $f$. Further, $K = F(\alpha_1, \ldots, \alpha_n)$ where the $\alpha_i$ are all roots of $f$ (in $L$).
\end{rem}


