\usepackage{graphicx}
\usepackage{amsmath}
\usepackage{amsthm}
\usepackage{amssymb}
\usepackage{marginnote}
%\input xy
%\xyoption{all}
\usepackage{pstricks, pst-node, pst-plot}
%\usepackage{MnSymbol}
\usepackage{mathrsfs}


\newcommand{\sidenote}[1]{\marginnote{\raggedright\footnotesize #1}}
\newcommand{\sidenotemark}[1]{${}^*$\marginnote{\raggedright\footnotesize\hspace{-0.3cm}$*$\hspace{0.1cm}#1}}

% Common Sets
\newcommand{\R}{\mathbb{R}}
\newcommand{\Rn}{\mathbb{R}^{n}}
\newcommand{\Rhat}{\widehat{\mathbb{R}}}
\newcommand{\C}{\mathbb{C}}
\newcommand{\N}{\mathbb{N}}
\newcommand{\Ccross}{\mathbb{C}^{\times}}
\newcommand{\Z}{\mathbb{Z}}
\newcommand{\Q}{\mathbb{Q}}
\newcommand{\Hp}{\mathcal{H}}
\newcommand{\cgp}{S^1}
\newcommand{\W}{\mathscr{W}}
\newcommand{\K}{\mathscr{K}}
\newcommand{\T}{\mathbb{T}}
\newcommand{\Ca}{\mathcal{C}}
\newcommand{\I}{\mathfrak{I}}
\newcommand{\RP}{\mathbb{RP}}
\newcommand{\CP}{\mathbb{CP}}
\newcommand{\Gr}{\mathrm{Gr}}
\newcommand{\Fi}{\mathbb{F}}
\newcommand{\Aid}{A_{\mathrm{inf\, div}}}
\newcommand{\Adiv}{A_{\mathrm{div}}}
\newcommand{\Mor}{\mathrm{Mor}}
\newcommand{\Cat}{\mathscr{C}}
\newcommand{\Dcat}{\mathscr{D}}
\newcommand{\Ob}{\mathrm{Ob}}
\newcommand{\Sets}{\mathbf{Sets}}
\newcommand{\Groups}{\mathbf{Groups}}
\newcommand{\Ab}{\mathbf{Ab}}
\newcommand{\VS}{\mathbf{VS}}
\newcommand{\Rings}{\mathbf{Rings}}
\newcommand{\Mod}[1]{#1\!\!-\!\!\mathbf{Mod}}
\newcommand{\rMod}[1]{\mathbf{Mod}\!\!-\!\!#1}
\newcommand{\Fields}{\mathbf{Fields}}
\newcommand{\Func}{\mathscr{F}}
\newcommand{\Field}{\mathbb{F}}
\newcommand{\Fq}{\mathbb{F}_q}

% Math operators
\DeclareMathOperator{\chr}{char}

% Spaces of Functions
\newcommand{\Sch}{\mathscr{S}}
\newcommand{\D}{\mathscr{D}}
\newcommand{\cinf}[1]{C^{\infty}(#1)}
\newcommand{\cinfzero}[1]{C^{\infty}_0 (#1)}
\newcommand{\cinfk}[1]{C^{\infty}_{K}(#1)}
\newcommand{\ck}[1]{C^{k}(#1)}
\newcommand{\cn}[2]{C^{#1}(#2)}
\newcommand{\cnk}[1]{C^{n}_{K}(#1)}
\newcommand{\czero}[1]{C^{0}(#1)}
\newcommand{\lspace}[2]{\mathcal{L}^{#1}({#2})}
\newcommand{\Lone}[1]{\lspace{1}{#1}}
\newcommand{\Ltwo}[1]{\lspace{2}{#1}}
\newcommand{\Lc}{L^2(\cgp)}
\newcommand{\lts}{\ell^{2,s}}
\newcommand{\ltms}{\ell^{2,-s}}

% Mappings/Transformations
\newcommand{\F}{\mathscr{F}}
\newcommand{\Fn}{\mathcal{F}}
\newcommand{\abs}[1]{\left|#1\right|}
\newcommand{\der}[1]{\frac{d}{d #1}}
\newcommand{\dern}[2]{\frac{d^{#1}}{d #2 ^{#1}}}
\newcommand{\dernum}[2]{\frac{d #1}{d #2}}
\newcommand{\pd}[1]{\frac{\partial}{\partial #1}}
\newcommand{\pdnum}[2]{\frac{\partial\!#1}{\partial\!#2}}
\newcommand{\pds}[2]{\frac{\partial^2 #1}{\partial #2^2}}
\newcommand{\pdn}[2]{\frac{\partial^{#1}}{\partial #2^{#1}}}
\newcommand{\mpd}[2]{\frac{\partial^2}{\partial #1 \partial #2}}
\newcommand{\ip}[2]{\left \langle #1, #2 \right \rangle}
\newcommand{\iip}[2]{\left\langle\!\langle #1, #2 \right\rangle\!\rangle}
\newcommand{\rint}[2]{\int_{\R} #1\, d#2}
\newcommand{\conv}{\ast}
\newcommand{\lieb}[2]{\left[#1, #2\right]}
\newcommand{\floor}[1]{\left\lfloor#1\right\rfloor}
\newcommand{\ceil}[1]{\left\lceil#1\right\rceil}
\newcommand{\diver}{\nabla \cdot}
\newcommand{\grad}{\nabla}
\newcommand{\curl}{\nabla \times}

% Misc Commands
\newcommand{\set}[1]{\left\{#1\right\}}
\newcommand{\modulus}[1]{\left|#1\right|}
\newcommand{\norm}[1]{\left\| #1\right\|}
\newcommand{\nomrl}[1]{\left\|#1\right\|_{L^2}}
\newcommand{\normck}[1]{\left\|#1\right\|_{C^k}}
\newcommand{\normn}[1]{\left\| #1\right\|_{n}}
\newcommand{\st}{\,:\,}
\newcommand{\supp}[1]{\mathrm{supp}(#1)}
\newcommand{\Hom}{\mathrm{Hom}}
\newcommand{\colim}[1]{\underset{#1}{\mathrm{colim\ }}}
\newcommand{\xyrightarrow}[2]{\underset{#2}{\xrightarrow{#1}}}
\newcommand{\poly}[2]{#1 \left[ #2 \right]}
\newcommand{\eoe}{\begin{flushright}\rightthumbsup\end{flushright}}
\newcommand{\lrp}[1]{\left( #1 \right)}
\newcommand{\lrsq}[1]{\left[ #1 \right]}
\newcommand{\Order}{\mathcal{O}}
\newcommand{\paren}[1]{\left(#1\right)}
\newcommand{\too}{\,\longrightarrow\,}
\newcommand{\Oh}{\mathcal{O}}
\newcommand{\ord}{\mathrm{ord}\,}
\newcommand{\e}{\varepsilon}
\newcommand{\im}{\mathrm{im}\,}
\newcommand{\angleb}[1]{\left\langle #1 \right\rangle}
\newcommand{\dangleb}[1]{\left\llangle #1 \right\rrangle}
\newcommand{\B}{\mathcal{B}}
\newcommand{\Aut}{\mathrm{Aut}\,}
\newcommand{\End}{\mathrm{End}\,}
\newcommand{\Inn}{\mathrm{Inn}\,}
\newcommand{\sqb}[1]{\left[ #1 \right]}
\newcommand{\A}{\mathcal{A}}
%\newcommand{\nto}{\nrightarrow}
\newcommand{\x}{\mathbf{x}}
\newcommand{\divides}{\,\big\vert\,}
\newcommand{\id}{\mathrm{id}}
\newcommand{\Ell}{\mathcal{L}}
\newcommand{\eval}[1]{\big\vert_{#1}}
\newcommand{\inv}{^{-1}}
\newcommand{\ican}{i_{\mathrm{can}}}
\newcommand{\inc}{\mathrm{inc}\,}
\newcommand{\tr}{\mathrm{tr}\,}
\newcommand{\sgn}{\mathrm{sgn}}
\newcommand{\rest}[1]{\restriction_{#1}}
\newcommand{\vspan}{\mathrm{span}\,}
\newcommand{\onto}{\twoheadrightarrow}
\renewcommand{\bar}[1]{\overline{#1}}
\newcommand{\codim}{\mathrm{codim}\,}
\newcommand{\Nil}{\mathrm{Nil}}
\newcommand{\lspan}{\mathrm{span}}
\newcommand{\HOT}{\mathrm{H.O.T.}}

%%%%%%%%%%%%%%%%%%%%%%%%%%%%%%%%%%%%%%%%%%%%%%%%%%%%%
% Symbols 
%%%%%%%%%%%%%%%%%%%%%%%%%%%%%%%%%%%%%%%%%%%%%%%%%%%%%

\DeclareMathSymbol{\transverse}{\mathrel}{AMSa}{"74}


%%%%%%%%%%%%%%%%%%%%%%%%%%%%%%%%%%%%%%%%%%%%%%%%%%%%%
% Theorem Styles
%%%%%%%%%%%%%%%%%%%%%%%%%%%%%%%%%%%%%%%%%%%%%%%%%%%%%

\newtheoremstyle{willthm}%		name
     {4pt}%      			Space above
     {4pt}%     	 		Space below
     {}%		    	        Body font
     {}%         			Indent amount (empty = no indent, \parindent = para indent)
     {\bfseries}% 			Thm head font
     {.}%        			Punctuation after thm head
     { }%	 			Space after thm head: " " = normal interword space; %       \newline = linebreak
     {}%         			Thm head spec (can be left empty, meaning `normal')

\theoremstyle{willthm}
\newtheorem{pr}{Problem}
\newtheorem{thm}{Theorem}
\newtheorem{lem}[thm]{Lemma}
\newtheorem{cor}[thm]{Corollary}
\newtheorem{prop}[thm]{Proposition}
\newtheorem{claim}[thm]{Claim}
\newtheorem{fact}[thm]{Fact}
\newtheorem{rem}[thm]{Remark}
\newtheorem{ex}[thm]{Exercise}
\newtheorem{eg}[thm]{Example}
\newtheorem{dfn}[thm]{Definition}
\newtheorem*{prs}{Problem}
\newtheorem*{thms}{Theorem}
\newtheorem*{lems}{Lemma}
\newtheorem*{cors}{Corollary}
\newtheorem*{props}{Proposition}
\newtheorem*{claims}{Claim}
\newtheorem*{facts}{Fact}
\newtheorem*{rems}{Remark}
\newtheorem*{exs}{Exercise}
\newtheorem*{notes}{Note}
\newtheorem*{egs}{Example}
\newtheorem*{dfns}{Definition}
\newtheorem*{ntns}{Notation}

%%%%%%%%%%%%%%%%%%%%%%%%%%%%%%%%%%%%%%%%%%%%%%%%%%%%%
% Environments
%%%%%%%%%%%%%%%%%%%%%%%%%%%%%%%%%%%%%%%%%%%%%%%%%%%%%

\newenvironment{ssprob}[1]
{
  \vspace{4pt}
  {\noindent\bfseries Stein \& Shakarchi problem #1. }
}
{\noindent}


\newenvironment{ssex}[1]
{
  \vspace{4pt}
  {\noindent\bfseries Stein \& Shakarchi exercise #1. }
}
{\noindent}

\newenvironment{gpex}[1]
{
  \vspace{4pt}
  {\noindent\bfseries Guillemin \& Pollack exercise #1.}
}
{\noindent}

\newenvironment{myenum}
{\begin{enumerate}[\indent (a)]}
{\end{enumerate}}

\newenvironment{sol}
{\begin{proof}[Solution.]}
{\end{proof}}

\newcommand{\smallvspace}{\vspace{4pt}}
