\section{Basic Definitions and Theorems}

\begin{dfn}
  A \emph{group} $G$ is a set along with a binary operation $\cdot : G \times G \to G$ satisfying
  \begin{myenum}
  \item Associativity: $(xy)z = x(yz)$ for all $x,y,z \in G$
  \item Identity: there exists $e \in G$ such that $e x = x e = x$ for all $x \in G$
  \item Inverses: for every $x \in G$ there exists $x^{-1} \in G$ such that $x x^{-1} = x^{-1} x = e$.
  \end{myenum}
  If $G$ additionally satisfies $xy = yx$ for all $x,y \in G$ we say that $G$ is \emph{abelian}.
\end{dfn}

The identity and inverses are unique. To see this, compute $e = e e' = e'$. Also, if $y$ and $y'$ are both inverses of $x$ then
\[
y = ye = y(xy') = (yx)y' = ey' = y'.
\]

\begin{eg}
  If $x^2 = e$ for all $x \in G$ then $G$ is abelian.
\end{eg}
\begin{proof}
  Let $x, y \in G$. Then we can write
  \[
  xy = x^{-1} y^{-1} = (yx)^{-1} = yx
  \]
  so $G$ is abelian.
\end{proof}
